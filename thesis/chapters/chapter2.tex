% @author Benjamin Schröder
%
\chapter{Grundlagen}
In diesem Kapitel werden die grundlegenden Konzepte vorgestellt, welche für das Verständnis der Arbeit benötigt werden.

\section{Prozedurale Generierung}
Prozedurale Generierung, oft auch Prozedurale Content Generierung (PCG), beschreibt eine Menge von Verfahren zum
algorithmischen Erstellen von Inhalten ("Content"). Dabei handelt es sich meist um Verfahren, die automatisch Texturen
oder verschiedene Gebilde im Kontext von Videospielen erzeugen können, so z.B. Landschaften, Flüsse, Straßennetze,
Städte oder Höhlenstrukturen. Auch Musik kann durch solche Verfahren generiert werden, was für diese Arbeit allerdings
weniger relevant ist.

Diese Definition ist absichtlich etwas allgemeiner gehalten, da das Aufstellen einer spezifischeren Definition nicht
besonders trivial ist. Das Konzept von PCG wurde bereits aus vielen veschiedenen Blickwinkeln beleuchtet und ist für verschiedene
Personen von unterschiedlicher Bedeutung. So hat z.B. ein Game Designer eine etwas andere Perspektive als ein Wissenschaftler, der
sich lediglich in der Theorie mit der Thematik beschäftigt. Verschiedene Definitionen unterscheiden sich in Bezug auf
Zufälligkeit, die Bedeutung von "Content", oder darin, ob und in welchem Umfang menschliche Intervenierung eine Rolle in einem
Verfahren spielen darf. Mit diesem Problem haben sich Togelius et al. \cite{9_togelius_et_al} bereits ausführlich befasst, weshalb dies hier
nicht weiter thematisiert werden soll. Für diese Arbeit soll die oben genannte Definition ausreichen.

\section{Verwendung von PCG}
% evtl. weglassen?
Da die Entwicklung von Videospielen aufgrund der großen Anzahl an benötigten Inhalten sehr schnell sehr aufwändig werden
kann, findet PCG vor allem in dieser Industrie einen großen Nutzen. Gerade das Erstellen von immersiven Welten erfordert eine Vielzahl
von verschiedensten detaillierten Modellen und kann manuell nur mit sehr großem Arbeitsaufwand umgesetzt werden. Das Automatisieren der
Generierung von Inhalten kann den Entwicklerstudios hier eine bedeutende Menge an Zeit und Kosten sparen, die dann an anderen
Stellen eingesetzt werden können. In vielen Fällen kann sogar Speicherplatz gespart werden, indem die Generierung der Inhalte
zur Laufzeit stattfindet.

PCG hat bereits in vielen bekannten Videospielen Verwendung gefunden. Schon im Jahr 1980 wurde