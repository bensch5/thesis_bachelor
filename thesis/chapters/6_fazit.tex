% @author Benjamin Schröder
%
\chapter{Fazit} % Auswerten der gesamten Arbeit
Abschließend soll diese Arbeit kurz reflektiert werden. Dazu fassen wir die gewonnenen Erkenntnisse zusammen und geben
anschließend einen Ausblick auf mögliche Verbesserungen und Erweiterungsmöglichkeiten des vorgestellten Ansatzes und
seiner Umsetzung.

\section{Zusammenfassung}
Im Rahmen dieser Arbeit wurde die inverse prozedurale Generierung von Strukturen für virtuelle Welten untersucht, dies spezifisch
für den zweidimensionalen Raum. Es wurden einige bereits entwickelte Ansätze im Bereich der ``normalen'' prozeduralen Generierung
vorgestellt und das vorhandene Kernproblem solcher Verfahren aufgezeigt. Es wurde klargestellt, wie hier durch das inverse Vorgehen
Abhilfe geschaffen werden kann, woraufhin einige der bereits existierenden inversen Verfahren erläutert wurden. Hier wurden
erneut die vorhandenen Probleme aufgezeigt und anschließend ein neues Verfahren präsentiert, welches versucht, auch diese Probleme
weitesgehend aus dem Weg zu räumen. Dieses Verfahren wurde detailliert vorgestellt und es wurde Schritt für Schritt erläutert,
wie wir es schaffen, Muster in einer vorgegebenen Struktur zu erkennen, diese in Form einer Grammatik abzubilden und abschließend
daraus beliebig viele Variationen abzuleiten, die dann für verschiedenste Zwecke genutzt werden können. Anhand einer prototypischen
Implementierung haben wir dann gezeigt, dass dieses Verfahren auch in der Praxis funktioniert, auch wenn darin noch einige Probleme
vorhanden sind.

\section{Verbesserungsansätze und Erweiterungsmöglichkeiten}
Das implementierte Verfahren sollte ausreichend demonstriert haben, was mithilfe des vorgestellten Ansatzes möglich ist. Dennoch gibt
es hier natürlich einige Verbesserungsansätze und Möglichkeiten zur Erweiterung der betrachteten Konzepte, welche wir gern abschließend
erwähnen möchten.

\subsection{Eingrenzen des Hierarchie-Wachstums}
In der momentanen Umsetzung ist es in den meisten Fällen nicht möglich, die Hierarchie wirklich zu leeren und diese wächst immer weiter
bis in die Unendlichkeit. Es ist uns nur möglich mit diesem Vorgehen zu einem Ende zu kommen, da wir das Erzeugen der Hierarchie
frühzeitig abbrechen, indem wir diese nur bis zu einer bestimmten Größe erweitern. Dadurch können wir zwar eine Vielzahl an Regeln finden,
nicht aber kann garantiert werden, dass wir mit der erzeugten Graphgrammatik dann alle zum Input lokal Ähnlichen Polygonstrukturen
erzeugen können. An dieser Stelle müsste das Verfahren um weitere Überprüfungen der erstellten Graphen erweitert werden, welche es uns
erlauben, diese frühzeitig auch ohne das Finden von Regeln aus der Hierarchie zu entfernen. Einige solcher Ansätze werden im Paper von
Merrell \cite{1_merrell} diskutiert, jedoch kann selbst durch diese nicht immer garantiert werden, dass die Hierarchie wirklich geleert
werden kann. \cite{1_merrell}

\subsection{Inkrementelles Festsetzen der Knotenpositionen}
Ein weiteres Problem mit dem hier vorgestellten Verfahren liegt darin, wie wir die aus der Graphgrammatik abgeleiteten Graphen in
eine feste geometrische Repräsentation bringen. Wie in \autoref{chap:auswertung} festgestellt wurde, läuft dies in unserer Implementierung
in vielen Fällen schief und es wird ein Output mit ungültigen Kantenlängen erzeugt. Auch bei vielen wiederholten Versuchen kann dieses
Problem meist nicht umgangen werden und letztendlich wird ein Output erzeugt, in welchem sich einige der Kanten überschneiden und teilweise
einen falschen Kantenwinkel besitzen. Ein Lösungsansatz hierfür wäre es, beim Festsetzen der Knotenpositionen inkrementell vorzugehen.
Aktuell erzeugen wir nur die Winkelgraphen selbst nach einem solchen Vorgehen, indem wir in jeder Iterationen nur eine neue Regel auf
den bereits erzeugten Graphen anwenden. Jedes Mal, wenn dann jedoch die Knotenpositionen für diesen Winkelgraphen bestimmt werden sollen,
fangen wir damit komplett von Neuem an und bestimmen alle Knotenpositionen auf einmal. Wenn dabei dann etwas schiefläuft, generieren wir
erneut alles von vorn. Würden wir stattdessen auch beim Bestimmen der Knotenpositionen inkrementell vorgehen und in jedem Schritt nur die
Positionen der in der aktuellen Iteration neu hinzugefügten Knoten berechnen, so würde das Verfahren deutlich seltener Probleme beim
Erzeugen des Outputs haben. Auch dies wird im Paper von Merrell diskutiert. \cite{1_merrell}

\subsection{Erweiterung in die dritte Dimension}
Die letzte der zu erwähnenden Verbesserungen ist die Erweiterung dieses Verfahrens in den dreidimensionalen Raum. Diese Arbeit hat sich
lediglich mit dem zweidimensionalen Raum befasst, da dies die Komplexität des Ganzen etwas verringert. Die grundlegenden vorgestellten
Konzepte können allerdings auch im 3D angewandt werden, sind dann nur etwas komplexer in der Umsetzung. Hierdurch würde das Verfahren
für viele weitere Anwendungsfälle in Frage kommen, die mit der aktuellen Umsetzung nicht abgedeckt werden können. Auch hierzu gibt es
weitere Informationen im Paper von Merrell. \cite{1_merrell}
