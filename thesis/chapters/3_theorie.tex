% @author Benjamin Schröder
%
\chapter{Theorie}
Im Folgenden werden die theoretischen Konzepte hinter dem praktischen Teil dieser Arbeit betrachtet. Das implementierte Verfahren wird
Schritt für Schritt vorgestellt und im Detail erläutert.

Konzepte:
- Graphen, Graph-Grammatiken, Graphersetzungssysteme (Erstellen von Regeln), Graphisomorphismen (Anwenden von Regeln)
- Local Similarity
- Einfärben von Facetten
- Facetten-Label
- Kanten-Label (gleiche anliegenden Farben, gleicher Tangentenwinkel)
- Kanten, Halbkanten
- Teilen (Cut-Operation) und Zusammenkleben (Branch & Loop Gluing) von Kanten
- Vollständige & unvollständige Graphen
- Planarität
- Positive & negative turns
- Graph Boundary String
- Einfachheit/Simplicity von Graphen
- Reduzierbarkeit (reducible graphs)
- Irreducible Graphs (alle Graphen sind entweder reduzierbar oder unvollständig)
- (Lösen von LGS zum Bestimmen der Knotenpositionen)

\section{Überblick}
- Beispielstruktur als Input (Graph)
- Aufteilen in Primitives (Teilen von Kanten in Halbkanten)
- Primitives in allen möglichen Wegen zusammenkleben zum Erstellen von Hierarchie
- Erstellen von Graph-Grammatik aus Hierarchie (im Idealfall können hiermit dann alle locally similar Graphen erstellt werden)
- Ableiten von neuen Graphen aus der Grammatik
- Festsetzen von Knotenpositionen des Graphen, um die letztendliche Geometrie zu erhalten
- Optional: Dekorieren/Texturieren der erhaltenen Geometrie (wahrscheinlich nicht direkt relevant für diese Arbeit)