% @author Benjamin Schröder
%
% Quellen: 5, 9, 10, 14, 15, 16, 17, 18, 19, 24, 25
%
% 5:
%   - Definition von PCG
%   - Vorstellen von einigen Verfahren
%
% 9:
%   - Definition von PCG durch Abgrenzung von anderen Verfahren
%
% 10:
%   - Geschichte und Verwendung von PCG
%   - Vorstellen einiger Verfahren
%
% 14:
%   - Vorstellen einiger Verfahren
%   - Unterscheidung zwischen assisted/non-assisted
%
% 15: The Death of the Level Designer
%   - kurze Definition von PCG
%   - Übersicht zu verschiedenen Anwendungsmöglichkeiten von PCG
%
% 16: An Image Synthesizer
%   - Perlin Noise
%
% 17: A Survey of Procedural Noise Functions
%   - Verwendung von Noise für PCG
%
% 18: Improving Noise
%   - Verbesserung von Perlin Noise
%
% 19: Fractals and the Geometry of Nature
%   - Fraktale (Mandelbrot)
%
% 24: A Very Short History of Dynamic and Procedural Content Generation
%   - Geschichte von PCG
%
% 25: Procedural Content Generation for Games: A Survey
%   - Klassifikation verschiedener Arten von PCG
%
% Weitere Links:
%   - https://en.wikipedia.org/wiki/Procedural_generation
%   Perlin Noise:
%   - https://mzucker.github.io/html/perlin-noise-math-faq.html#whatsnoise
%   - https://adrianb.io/2014/08/09/perlinnoise.html
%   - https://mrl.cs.nyu.edu/~perlin/doc/oscar.html
%   Simplex Noise:
%   - https://thebookofshaders.com/11/?lan=de

\chapter{Grundlagen}
Als Grundlage für das Verständnis des weiteren Inhalts dieser Arbeit machen wir zunächst einen kurzen Abstecher in den
Bereich der prozeduralen Generierung allgemein. Wir stellen klar, was unter diesem Begriff zu verstehen ist und widmen
uns außerdem kurz der zugrundeliegenden Geschichte. Dabei werden wir einige fundamentale Errungenschaften und Verfahren
betrachten, die den Weg zum aktuellen Forschungsstand geprägt haben.

\section{Prozedurale Generierung}
Prozedurale Generierung, oder auch \gls{ac:pcg}, beschreibt eine Menge von Verfahren zum algorithmischen Erstellen von
Inhalten (``Content''). Dabei handelt es sich meist um Inhalte in Form von Texturen oder verschiedenen Gebilden im Kontext
von Videospielen und anderen Simulationen, wie z.B. Landschaften, Flüsse, Straßennetze, Städte oder Höhlenstrukturen.
\cite{14_carli_et_al} Auch Musik kann durch solche Verfahren generiert werden. \cite{28_ramanto_maulidevi}

Diese Definition ist absichtlich etwas allgemeiner gehalten, da das Aufstellen einer spezifischeren Definition nicht
besonders trivial ist. Das Konzept von \gls{ac:pcg} wurde bereits aus vielen veschiedenen Blickwinkeln beleuchtet und
ist für verschiedene Personen von unterschiedlicher Bedeutung. So hat z.B. ein Game Designer eine etwas andere Perspektive
als ein Wissenschaftler, der sich lediglich in der Theorie mit der Thematik beschäftigt. \cite{9_togelius_et_al}
Verschiedene Definitionen unterscheiden sich in Bezug auf Zufälligkeit, die Bedeutung von ``Content'', oder darin, ob und
in welchem Umfang menschliche Intervenierung eine Rolle in einem Verfahren spielen darf. Smelik et al. definieren ``Content''
als jegliche Art von automatisch generierten Inhalten, welche irgendwie anhand von einer begrenzten Menge an Nutzer-definierten
Parametern erzeugt werden können. \cite{26_smelik_et_al} Timothy Roden und Ian Parberry beschreiben entsprechende Verfahren zur
Erzeugung dieser Inhalte als \textit{Vermehrungsalgorithmen} (``amplification algorithms''), da diese eine kleinere Menge
von Inputparametern entgegennehmen und diese in eine größere Menge an Outputdaten transformieren. \cite{27_roden_parberry}
Togelius et al. \cite{9_togelius_et_al} versuchen den Bereich genauer abzugrenzen, indem sie anhand von Gegenbeispielen
aufzeigen, was \textit{nicht} als \gls{ac:pcg} bezeichnet werden sollte. So zählt für Togelius et al. z.B. das Erstellen von
Inhalten eines Videospiels mittels Level-Editor in keinem Fall als \gls{ac:pcg}, auch wenn dabei das Spiel indirekt durch z.B.
automatisches Hinzufügen oder Anpassen von Strukturen beeinflusst wird. Generell wird sich in der Arbeit
von Togelius et al. \cite{9_togelius_et_al} ausführlich mit dem Problem der Definition von \gls{ac:pcg} befasst, weshalb
dies hier nun nicht weiter thematisiert werden soll. Die oben genannte grobe Erklärung fasst die Kernaussage der verschiedenen
Definitionen weitesgehend zusammen und sollte für unsere Zwecke ausreichen.

\section{Geschichte von PCG}
