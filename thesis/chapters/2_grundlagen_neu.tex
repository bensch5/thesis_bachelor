% @author Benjamin Schröder
%
% Quellen: 5, 9, 10, 14, 15, 16, 17, 18, 19, 24, 25
%
% 5:
%   - Definition von PCG
%   - Vorstellen von einigen Verfahren
%
% 9:
%   - Definition von PCG durch Abgrenzung von anderen Verfahren
%
% 10:
%   - Geschichte und Verwendung von PCG
%   - Vorstellen einiger Verfahren
%
% 14:
%   - Vorstellen einiger Verfahren
%   - Unterscheidung zwischen assisted/non-assisted
%
% 15: The Death of the Level Designer
%   - kurze Definition von PCG
%   - Übersicht zu verschiedenen Anwendungsmöglichkeiten von PCG
%
% 16: An Image Synthesizer
%   - Perlin Noise
%
% 17: A Survey of Procedural Noise Functions
%   - Verwendung von Noise für PCG
%
% 18: Improving Noise
%   - Verbesserung von Perlin Noise
%
% 19: Fractals and the Geometry of Nature
%   - Fraktale (Mandelbrot)
%
% 24: A Very Short History of Dynamic and Procedural Content Generation
%   - Geschichte von PCG
%
% 25: Procedural Content Generation for Games: A Survey
%   - Klassifikation verschiedener Arten von PCG
%
% Weitere Links:
%   - https://en.wikipedia.org/wiki/Procedural_generation
%   Perlin Noise:
%   - https://mzucker.github.io/html/perlin-noise-math-faq.html#whatsnoise
%   - https://adrianb.io/2014/08/09/perlinnoise.html
%   - https://mrl.cs.nyu.edu/~perlin/doc/oscar.html
%   Simplex Noise:
%   - https://thebookofshaders.com/11/?lan=de

\chapter{Grundlagen}
Als Grundlage für das Verständnis des weiteren Inhalts dieser Arbeit machen wir zunächst einen kurzen Abstecher in den
Bereich der prozeduralen Generierung allgemein. Wir stellen klar, was unter diesem Begriff zu verstehen ist und widmen
uns außerdem kurz der zugrundeliegenden Geschichte. Dabei werden einige fundamentale Errungenschaften und Verfahren
betrachtet, die den Weg zum momentanen Forschungsstand geprägt haben.

\section{Prozedurale Generierung}
Prozedurale Generierung, oft auch \gls{ac:pcg}, beschreibt eine Menge von Verfahren zum
algorithmischen Erstellen von Inhalten ("Content"). Dabei handelt es sich meist um Verfahren, die automatisch Texturen
oder verschiedene Gebilde im Kontext von Videospielen erzeugen können, so z.B. Landschaften, Flüsse, Straßennetze,
Städte oder Höhlenstrukturen. Auch Musik kann durch solche Verfahren generiert werden, was für diese Arbeit allerdings
weniger relevant ist. \cite{9_togelius_et_al}

\section{Geschichte von PCG}
