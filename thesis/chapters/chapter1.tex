% @author Benjamin Schröder
%
\chapter{Einleitung}
\section{Motivation}
In diesem Abschnitt wird erklärt, wieso die prozedurale Generierung überhaupt so ein wichtiges Thema ist.
Es wird geklärt, wer davon Gebrauch macht, und wieso es für den entsprechenden Anwender Sinn macht. Dazu zählt
zum Einen das Einsparen von Ressourcen, aber auch das Umsetzen von Spielkonzepten, die durch die hier vorgestellten
Verfahren erst möglich werden.

\section{Problemstellung}
Hier wird dann darauf aufmerksam gemacht, dass es bei diesen Verfahren viele Limitationen gibt. Bei vielen Verfahren
ist es nötig, manuell Regeln für den Algorithmus zu erstellen, sodass dieser überhaupt arbeiten kann. Dies setzt wiederum
einiges an Kenntnissen voraus und ist somit nicht für jeden zugänglich. Außerdem werden weitere Probleme aufgezeigt.

\section{Ziele und Vorgehen}
Aus den aufgezeigten Problemen ergibt sich nun der Sinn dieser Arbeit. Inverse Verfahren beheben die oben genannten Probleme
und sollen deswegen genauer untersucht werden. Es werden verschiedene Verfahren analysiert und miteinander verglichen.
Anschließend wird ein entsprechendes und vielversprechendes Verfahren im Detail untersucht, theoretisch erläutert und
dann prototypisch implementiert.