% @author Benjamin Schröder
%
\chapter{Einleitung}

\section{Motivation}
% In diesem Abschnitt wird erklärt, wieso die prozedurale Generierung überhaupt so ein wichtiges Thema ist.
% Es wird geklärt, wer davon Gebrauch macht, und wieso es für den entsprechenden Anwender Sinn macht. Dazu zählt
% zum Einen das Einsparen von Ressourcen, aber auch das Umsetzen von Spielkonzepten, die durch die hier vorgestellten
% Verfahren erst möglich werden.
Die Erstellung von fiktiven Welten spielt eine große Rolle in vielen Videospielen, Filmen, Virtual Reality Umgebungen
und weiteren Bereichen der Simulation. Hierfür wird eine Vielzahl an verschiedenen Objekten und Strukturen benötigt, um
ein nicht-repetitives und immersives Erlebnis für den Endnutzer zu schaffen. All dies manuell anzufertigen, stellt vor
allem kleinere Indie-Entwicklerstudios vor eine große Herausforderung und kann die Entwicklungszeit signifikant in die
Länge ziehen. Selbst in größeren Teams mit einer Vielzahl von Designern nimmt die Erstellung von realistischen Welten zumindest
einige Monate in Anspruch. \cite{10_freiknecht} Hier kann an vielen Stellen nachgeholfen werden, indem man das Erstellen
von Inhalten automatisiert. Entsprechende Prozesse lassen sich dem Bereich der prozeduralen Generierung zuordnen.
% TODO: Konkretisieren und mehr verschiedene Quellen angeben, Ausgangspunkt: Quelle 25


Mithilfe von verschiedensten Verfahren können so z.B. einzelne Dungeons oder sogar ganze Welten und darin enthaltene Gebilde
automatisch erzeugt werden. Diese bilden eine Grundstruktur für ein komplexeres Design, bei dem die Entwickler dann nur noch
kleinere Details per Hand abändern oder hinzufügen müssen. \cite{10_freiknecht}

Andererseits existieren auch viele Videospiele, wie z.B. Minecraft\footnote{https://www.minecraft.net/} oder
Terraria\footnote{https://terraria.org/}, die auf prozeduraler Generierung aufbauen, um ihr Spielkonzept umzusetzen.
Konkret wird einem neuen Spieler hier eine komplett neue und einzigartige, aber dennoch logisch
zusammenhängende Welt generiert. Somit macht jeder Spieler eine andere Erfahrung und kann das Spiel außerdem gewissermaßen
unbegrenzt oft durchspielen, ohne dass es repetitiv wirkt. So etwas wäre ohne Automatisierung gar nicht erst umsetzbar.

\section{Problemstellung}
% Hier wird dann darauf aufmerksam gemacht, dass es bei diesen Verfahren viele Limitationen gibt. Bei vielen Verfahren
% ist es nötig, manuell Regeln für den Algorithmus zu erstellen, sodass dieser überhaupt arbeiten kann. Dies setzt wiederum
% einiges an Kenntnissen voraus und ist somit nicht für jeden zugänglich. Außerdem werden weitere Probleme aufgezeigt.

% Alte Formulierung:
% Es gibt viele bekannte Verfahren, welche solche Ergebnisse unter der Verwendung von u.a. zellulären Automaten, generativen
% Grammatiken oder Constraint-basierten Graphen erzielen können. \cite{5_van_der_linden_et_al} Ein Großteil dieser Verfahren erfordert jedoch
% menschliches Eingreifen in einigen der Teilschritte. So z.B. muss beim Verwenden einer generativen Grammatik meist bereits eine Menge
% an Produktionsregeln durch einen Menschen vorgegeben werden, bevor die automatische Generierung überhaupt beginnen kann. Das Erstellen
% solcher Regeln ist mit viel Arbeit und Trial-and-Error verbunden und kann ohne ein ausgeprägtes Verständnis des angewandten Verfahrens
% sehr schwierig werden. Dadurch kommt es für viele Designer letztendlich doch nicht in Frage. Auch gibt es Szenarien, in denen die Generierung
% von Inhalten durch den Endnutzer beeinflusst werden kann, so z.B. in Spielen, in denen der Spieler dynamisch mit dem Terrain und anderen
% Strukturen interagieren kann. In einem solchen Fall kann der Entwickler keinen direkten Einfluss auf den Generierungsprozess nehmen und alles
% muss voll automatisiert sein. \cite{14_carli_et_al} Hier setzt diese Arbeit an und untersucht Möglichkeiten zur vollständigen Automatisierung
% solcher Verfahren.

Es gibt viele bekannte Verfahren, welche solche Ergebnisse unter der Verwendung von u.a. zellulären Automaten, generativen
Grammatiken oder Constraint-basierten Graphen erzielen können. \cite{5_van_der_linden_et_al} Ein Großteil dieser Verfahren erfordert
menschliches Eingreifen in einige der Teilschritte. Es gibt allerdings Szenarien, in denen dies problematisch wird. Hängt die Generierung von
Inhalten eines Produkts z.B. von Entscheidungen des Endnutzers ab (z.B. in Spielen, in denen der Spieler dynamisch mit
dem Terrain und anderen Strukturen interagiert), so kann der Entwickler keinen direkten Einfluss auf den Generierungsprozess
nehmen und alles muss voll automatisiert sein. \cite{14_carli_et_al} Auch in Projekten, in denen dies nicht der Fall ist und der gesamte Inhalt
im Voraus erstellt wird, kann das Voraussetzen von menschlicher Intervenierung als Teil des Prozesses zu einem Problem werden.
Ein Beispiel hierfür wären Verfahren, die eine generative Grammatik nutzen und voraussetzen, dass dafür zunächst eine Menge an Produktionsregeln
durch einen Menschen vorgegeben werden (bspw. \cite{33_adams}), bevor die automatische Generierung überhaupt beginnen kann. Das Erstellen solcher
Regeln ist mit viel Arbeit und Trial-and-Error verbunden und kann ohne ein ausgeprägtes Verständnis des angewandten Verfahrens sehr schwierig werden.
Dadurch wird der Einsatz eines solchen Verfahrens für viele Designer letztendlich nicht in Frage kommen.
Hier setzt diese Arbeit an und untersucht Möglichkeiten zur Automatisierung des Erstellens solcher Regeln.

\section{Ziele und Vorgehen}
% Aus den aufgezeigten Problemen ergibt sich nun der Sinn dieser Arbeit. Inverse Verfahren beheben die oben genannten Probleme
% und sollen deswegen genauer untersucht werden. Es werden verschiedene Verfahren analysiert und miteinander verglichen.
% Anschließend wird ein entsprechendes und vielversprechendes Verfahren im Detail untersucht, theoretisch erläutert und
% dann prototypisch implementiert.
Spezifisch soll versucht werden, Muster in Beispielstrukturen zu identifizieren. Aus diesen Mustern sollen dann Regeln zum Zusammensetzen
von Strukturen mit ähnlichen Eigenschaften abgeleitet werden. Gelingt dies, so muss ein Designer lediglich ein einziges Beispielmodell erstellen,
um damit seine kreative Vision abzubilden. Alle weiteren Schritte zum Ableiten von Variationen dieses Inputs laufen anschließend automatisch ab.
Dies nennen wir \textit{inverse} prozedurale Generierung,
da der Prozess mit einem soweit fertigen Modell beginnt und daraus dann die Regeln ableitet, statt wie in den klassischeren Verfahren zuerst mit
der Erstellung der Regeln zu beginnen. Die Erstellung eines Beispielmodells erfordert zwar nach wie vor die Arbeit eines Designers, anschließend
ist aber kein menschliches Eingreifen mehr nötig und das eigentliche Verfahren läuft vollautomatisch ab.

Es gibt bereits verschiedene Verfahren, die einen solchen Ansatz verfolgen. Diese sind u.a. der Gitter-basierte Wave Function
Collapse Algorithmus von Maxim Gumin\footnote{https://github.com/mxgmn/WaveFunctionCollapse/}, die nach Symmetrien suchende inverse
prozedurale Modellierung von Bokeloh et al. \cite{3_bokeloh_et_al}, oder das Polygon-basierte Verfahren von Paul Merrell. \cite{1_merrell}

Im Rahmen dieser Arbeit werden jene Verfahren grob analysiert und anschließend das vielversprechendste davon praktisch umgesetzt. Das Endergebnis
der Arbeit soll dann sein, dass die Funktionsweise eines ausgewählten Konzepts ausführlich und verständlich dargestellt, und nach eigener
Interpretation konkret implementiert wird. Wir begrenzen uns dabei auf die Generierung von Strukturen im zweidimensionalen Raum. Die genauen
Anforderungen an das darzustellende Konzept sowie an die umgesetzte Software werden in den dazugehörigen Kapiteln näher erläutert.