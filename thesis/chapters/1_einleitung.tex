% @author Benjamin Schröder
%
\chapter{Einleitung}

\section{Motivation}
% In diesem Abschnitt wird erklärt, wieso die prozedurale Generierung überhaupt so ein wichtiges Thema ist.
% Es wird geklärt, wer davon Gebrauch macht, und wieso es für den entsprechenden Anwender Sinn macht. Dazu zählt
% zum Einen das Einsparen von Ressourcen, aber auch das Umsetzen von Spielkonzepten, die durch die hier vorgestellten
% Verfahren erst möglich werden.
Die Erstellung von fiktiven Welten spielt eine große Rolle in vielen Videospielen, Filmen, Virtual Reality Umgebungen
und weiteren Bereichen der Simulation. Hierfür wird eine Vielzahl an verschiedenen Objekten und Strukturen benötigt, um
ein nicht-repetitives und immersives Erlebnis für den Endnutzer zu schaffen. All dies manuell anzufertigen, stellt vor
allem kleinere Indie-Entwicklerstudios vor eine große Herausforderung und kann die Entwicklungszeit signifikant in die
Länge ziehen. Selbst in größeren Teams mit einer Vielzahl von Designern, nimmt die Erstellung von realistischen Welten
einige Monate in Anspruch. \cite{10_freiknecht} Hier kann an vielen Stellen nachgeholfen werden, indem man das Erstellen
von Inhalten automatisiert. Entsprechende Prozesse lassen sich dem Bereich der prozeduralen Generierung zuordnen.

Mithilfe von verschiedensten Verfahren können so z.B. einzelne Dungeons oder sogar ganze Welten und darin enthaltene Gebilde
automatisch erzeugt werden. Diese bilden eine Grundstruktur für ein komplexeres Design, bei dem die Entwickler dann nur noch
kleinere Details per Hand abändern oder hinzufügen müssen.

Andererseits existieren auch viele Videospiele, wie z.B. Minecraft\footnote{https://www.minecraft.net/} oder
Terraria\footnote{https://terraria.org/}, die auf prozeduraler Generierung aufbauen, um ihr Spielkonzept umzusetzen.
Konkret wird einem neuen Spieler hier eine komplett neue und einzigartige, aber dennoch logisch
zusammenhängende Welt generiert. Somit macht jeder Spieler eine andere Erfahrung und kann das Spiel außerdem gewissermaßen
unbegrenzt oft durchspielen, ohne dass es repetitiv wirkt. So etwas wäre ohne Automatisierung gar nicht erst umsetzbar.

\section{Problemstellung}
% Hier wird dann darauf aufmerksam gemacht, dass es bei diesen Verfahren viele Limitationen gibt. Bei vielen Verfahren
% ist es nötig, manuell Regeln für den Algorithmus zu erstellen, sodass dieser überhaupt arbeiten kann. Dies setzt wiederum
% einiges an Kenntnissen voraus und ist somit nicht für jeden zugänglich. Außerdem werden weitere Probleme aufgezeigt.
Es gibt viele bekannte Verfahren, welche solche Ergebnisse unter der Verwendung von u.a. zellulären Automaten, generativen
Grammatiken oder Constraint-basierten Graphen erzielen können. \cite{5_van_der_linden_et_al} Größtenteils beruhen diese jedoch auf der Anwendung
von manuell erstellten Regeln, so z.B. eine Menge an gegebenen Produktionsregeln bei der Nutzung von Grammatiken. Das Erstellen
solcher Regeln ist mit viel Arbeit und Trial-and-Error verbunden und kann ohne ein ausgeprägtes Verständnis des angewandten Verfahrens
sehr schwierig werden. Dadurch kommt es für viele Designer letztendlich doch nicht in Frage. Hier setzt diese Arbeit an und untersucht
die automatische Erstellung solcher Regeln.

\section{Ziele und Vorgehen}
% Aus den aufgezeigten Problemen ergibt sich nun der Sinn dieser Arbeit. Inverse Verfahren beheben die oben genannten Probleme
% und sollen deswegen genauer untersucht werden. Es werden verschiedene Verfahren analysiert und miteinander verglichen.
% Anschließend wird ein entsprechendes und vielversprechendes Verfahren im Detail untersucht, theoretisch erläutert und
% dann prototypisch implementiert.
Spezifisch soll versucht werden, Muster in Beispielstrukturen zu erkennen. Aus diesen Mustern sollen dann Regeln zum Zusammensetzen
von Strukturen mit ähnlichen Eigenschaften abgeleitet werden.

Hier gibt es bereits verschiedene Verfahren, die einen solchen Ansatz verfolgen. Diese sind u.a. der Gitter-basierte Wave Function
Collapse Algorithmus von Maxim Gumin\footnote{https://github.com/mxgmn/WaveFunctionCollapse/}, die nach Symmetrien suchende inverse
prozedurale Modellierung von Bokeloh et al. \cite{3_bokeloh_et_al}, oder das Polygon-basierte Verfahren von Paul Merrell. \cite{1_merrell}

Diese Verfahren werden analysiert und anschließend das vielversprechendste davon praktisch umgesetzt. Das Endergebnis der Arbeit
soll dann sein, dass das ausgewählte Konzept ausführlich und verständlich erläutert, und nach eigener Interpretation konkret
implementiert wird. Im Rahmen dieser Arbeit soll dies lediglich für den zweidimensionalen Raum geschehen, könnte jedoch im Anschluss
auch auf die dritte Dimension ausgeweitet werden.

Ebenfalls soll eine grafische Benutzeroberfläche bereitgestellt werden, über welche der Endnutzer Inputstrukturen auswählen, sowie
Parameter zur Beeinflussung des Algorithmus anpassen kann.