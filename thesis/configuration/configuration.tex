% !TEX root = ../thesis.tex
%
% configurations
%

% English Language support
% -> uncomment if needed
% Beta!
%\fullenglish{yes}
\fullenglish{no}

% text field
%-> replace supervisor names with correct ones
\firstSupervisor{Prof. Dr. Philipp Jenke}
\secondSupervisor{Prof. Dr. Peer Stelldinger}

% text field
%-> replace title with your thesis title
\thesisTitle{Beispiel-basierte inverse prozedurale Generierung für zweidimensionale Szenen}
\thesisTitleEN{Example-based inverse procedural generation for two-dimensional scenes}

% text field
%-> replace the key words with your own key words
\keywordsDE{2D, Prozeduale Generierung, Inverse prozedurale Generierung, Ableiten einer Graphgrammatik, Polygone}
\keywordsEN{2D, procedural generation, inverse procedural generation, deriving a graph grammar, polygons}

% text field
%-> replace the text with a description of the thesis
\abstractDE{
    Die Erstellung von immersiven fiktiven Welten in Videospielen und anderen Simulationen erfordert einen großen manuellen
    Aufwand der Entwickler, welcher oftmals mit viel monotoner Modellierungsarbeit einhergeht. Um diesen Aufwand zu minimieren, wurden über
    die Jahre viele Verfahren entwickelt, die durch das Automatisieren entsprechender Aufgaben Abhilfe schaffen sollen. Solche
    Verfahren fallen in den Bereich der prozeduralen Generierung. Hierbei gibt es eine Vielzahl verschiedene Ansätze, von denen einige
    in dieser Arbeit betrachtet werden. Viele dieser Verfahren sind jedoch nur stark eingeschränkt anwendbar oder erfordern
    das manuelle Bereitstellen von Regeln, wofür oft ein tiefes Verständnis des entsprechenden Verfahrens vorausgesetzt wird.
    In dieser Arbeit wird eine bestimmte Art von prozeduralen Generierungsverfahren genauer beleuchtet, welche auch diese Mängel
    beseitigen soll, indem solche Regeln automatisch aus einer gegebenen Beispielstruktur abgeleitet werden. Nach einem kurzen
    Vergleich wird eines solcher Verfahren im Detail erläutert und prototypisch umgesetzt.
}

\abstractEN{
    The creation of immersive fictional worlds in video games and other simulations requires a great deal of manual effort on the
    part of developers, often accompanied by a lot of monotonous modelling work. To minimize this effort, many methods have been developed
    over the years to automate these tasks. Such methods fall under the umbrella of procedural generation. There are many such
    approaches, some of which are discussed in this paper. However, many of these procedures are very limited in their applicability
    or require the manual provision of rules, which often requires a deep understanding of the corresponding procedure. This paper
    examines a specific type of procedural generation methods that aims to overcome these shortcomings by automatically deriving such
    rules from a given example structure. After a short comparison, one such procedure is explained in detail and implemented as a prototype.
}

% text field
%-> replace john with your name
\thesisAuthor{Benjamin Schröder}

% text field
%-> enter the submission date
\submissionDate{11. Juli 2024}

% switch - uncomment only one
%-> uncomment NDA or public
%\NDA{yes}
\NDA{no}

% switch - uncomment only one
%-> uncomment old standard cover or cover Corporate Design 2017
\Cover{CD2017}
%\Cover{CD2017NoLogo}
%\Cover{Std2018}
%\Cover{Std2018_green} 			% with green bar

% switch - uncomment only one
%-> uncomment to show list of figures or not
\ListOfFigures{yes}
%\ListOfFigures{no}

% switch - uncomment only one
%-> uncomment to show list of tables or not
\ListOfTables{yes}
%\ListOfTables{no}

% switch - uncomment only one
%-> uncomment to show list of accronyms or not
\ListOfAccronyms{yes}
%\ListOfAccronyms{no}

% switch - uncomment only one
%-> uncomment to show list of symbols or not
\ListOfSymbols{yes}
%\ListOfSymbols{no}

% switch - uncomment only one
%-> uncomment to show list of glossary entries or not
\Glossary{yes}
%\Glossary{no}

% switch - uncomment only one
%-> uncomment the study course you are in
%\studycourse{ITS}
%\studycourse{TI}
\studycourse{AI}
%\studycourse{WI}
%\studycourse{EI}
%\studycourse{REE}
%\studycourse{BMP}		
%\studycourse{BMP-hp}	 % Internship Report in M&P
%\studycourse{BMT}
%\studycourse{BMT-st}    % Study / home assignment in BMT
%\studycourse{BMT-hp}    % Internship Report in BMT
%\studycourse{MI}
%\studycourse{MIK}
%\studycourse{MA}

\def\imgHeight{100pt}
\def\imgWidth{410pt}
